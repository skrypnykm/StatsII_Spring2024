\documentclass[12pt,letterpaper]{article}
\usepackage{graphicx,textcomp}
\usepackage{natbib}
\usepackage{setspace}
\usepackage{fullpage}
\usepackage{color}
\usepackage[reqno]{amsmath}
\usepackage{amsthm}
\usepackage{fancyvrb}
\usepackage{amssymb,enumerate}
\usepackage[all]{xy}
\usepackage{endnotes}
\usepackage{lscape}
\newtheorem{com}{Comment}
\usepackage{float}
\usepackage{hyperref}
\newtheorem{lem} {Lemma}
\newtheorem{prop}{Proposition}
\newtheorem{thm}{Theorem}
\newtheorem{defn}{Definition}
\newtheorem{cor}{Corollary}
\newtheorem{obs}{Observation}
\usepackage[compact]{titlesec}
\usepackage{dcolumn}
\usepackage{tikz}
\usetikzlibrary{arrows}
\usepackage{multirow}
\usepackage{xcolor}
\newcolumntype{.}{D{.}{.}{-1}}
\newcolumntype{d}[1]{D{.}{.}{#1}}
\definecolor{light-gray}{gray}{0.65}
\usepackage{url}
\usepackage{listings}
\usepackage{color}

\definecolor{codegreen}{rgb}{0,0.6,0}
\definecolor{codegray}{rgb}{0.5,0.5,0.5}
\definecolor{codepurple}{rgb}{0.58,0,0.82}
\definecolor{backcolour}{rgb}{0.95,0.95,0.92}

\lstdefinestyle{mystyle}{
	backgroundcolor=\color{backcolour},   
	commentstyle=\color{codegreen},
	keywordstyle=\color{magenta},
	numberstyle=\tiny\color{codegray},
	stringstyle=\color{codepurple},
	basicstyle=\footnotesize,
	breakatwhitespace=false,         
	breaklines=true,                 
	captionpos=b,                    
	keepspaces=true,                 
	numbers=left,                    
	numbersep=5pt,                  
	showspaces=false,                
	showstringspaces=false,
	showtabs=false,                  
	tabsize=2
}
\lstset{style=mystyle}
\newcommand{\Sref}[1]{Section~\ref{#1}}
\newtheorem{hyp}{Hypothesis}

\title{Problem Set 2}
\date{Due: February 18, 2024}
\author{Applied Stats II \\ \vspace{\baselineskip}
	\textbf{Maiia Skrypnyk 23371609}}

\begin{document}
	\maketitle
	\section*{Instructions}
	\begin{itemize}
		\item Please show your work! You may lose points by simply writing in the answer. If the problem requires you to execute commands in \texttt{R}, please include the code you used to get your answers. Please also include the \texttt{.R} file that contains your code. If you are not sure if work needs to be shown for a particular problem, please ask.
		\item Your homework should be submitted electronically on GitHub in \texttt{.pdf} form.
		\item This problem set is due before 23:59 on Sunday February 18, 2024. No late assignments will be accepted.
	%	\item Total available points for this homework is 80.
	\end{itemize}

	
	%	\vspace{.25cm}
	
%\noindent In this problem set, you will run several regressions and create an add variable plot (see the lecture slides) in \texttt{R} using the \texttt{incumbents\_subset.csv} dataset. Include all of your code.

	\vspace{.25cm}
%\section*{Question 1} %(20 points)}
%\vspace{.25cm}
\noindent We're interested in what types of international environmental agreements or policies people support (\href{https://www.pnas.org/content/110/34/13763}{Bechtel and Scheve 2013)}. So, we asked 8,500 individuals whether they support a given policy, and for each participant, we vary the (1) number of countries that participate in the international agreement and (2) sanctions for not following the agreement. \\

\noindent Load in the data labeled \texttt{climateSupport.RData} on GitHub, which contains an observational study of 8,500 observations.

\begin{itemize}
	\item
	Response variable: 
	\begin{itemize}
		\item \texttt{choice}: 1 if the individual agreed with the policy; 0 if the individual did not support the policy
	\end{itemize}
	\item
	Explanatory variables: 
	\begin{itemize}
		\item
		\texttt{countries}: Number of participating countries [20 of 192; 80 of 192; 160 of 192]
		\item
		\texttt{sanctions}: Sanctions for missing emission reduction targets [None, 5\%, 15\%, and 20\% of the monthly household costs given 2\% GDP growth]
		
	\end{itemize}
	
\end{itemize}

\newpage
\noindent Please answer the following questions:

\begin{enumerate}
	\item
	Remember, we are interested in predicting the likelihood of an individual supporting a policy based on the number of countries participating and the possible sanctions for non-compliance.
	\begin{enumerate}
		\item [] Fit an additive model. Provide the summary output, the global null hypothesis, and $p$-value. Please describe the results and provide a conclusion.

\vspace{.25cm}
\textbf{Preparing the data:}
	\lstinputlisting[language=R, firstline=39, lastline=46]{PS2MS.R} 
	
	
\textbf{Logistic regression formula:}

\begin{center}
	$\text{logit}(P(\text{choice}=1)) = \beta_0 + \beta_1 X_{\text{countries80}} + \beta_2 X_{\text{countries160}} + \beta_3 X_{\text{sanctions}5\%} + \beta_4 X_{\text{sanctions}15\%} + \beta_5 X_{\text{sanctions}20\%}$
\end{center}


\textbf{Summary output:}

	\lstinputlisting[language=R, firstline=48, lastline=52]{PS2MS.R} 
	
\textit{	(Please see the summary table on the next page)}

	\begin{table}[H] \centering   \caption{}   \label{} \begin{tabular}{@{\extracolsep{5pt}}lc} \\[-1.8ex]\hline \hline \\[-1.8ex]  & \multicolumn{1}{c}{\textit{Dependent variable:}} \\ \cline{2-2} \\[-1.8ex] & choice \\ \hline \\[-1.8ex]  countries80 of 192 & 0.33636$^{***}$ \\   & (0.054) \\   & \\  countries160 of 192 & 0.64835$^{***}$ \\   & (0.054) \\   & \\  sanctions5\% & 0.19186$^{***}$ \\   & (0.062) \\   & \\  sanctions15\% & $-$0.13325$^{**}$ \\   & (0.062) \\   & \\  sanctions20\% & $-$0.30356$^{***}$ \\   & (0.062) \\   & \\  Constant & $-$0.27266$^{***}$ \\   & (0.054) \\   & \\ \hline \\[-1.8ex] Observations & 8,500 \\ Log Likelihood & $-$5,784.130 \\ Akaike Inf. Crit. & 11,580.260 \\ \hline \hline \\[-1.8ex] \textit{Note:}  & \multicolumn{1}{r}{$^{*}$p$<$0.1; $^{**}$p$<$0.05; $^{***}$p$<$0.01} \\ \end{tabular} \end{table} 

\vspace{.25cm}
\textbf{Global null hypothesis:}

$H_0: \beta_1 = \beta_2 = \beta_3 = \beta_4 = \beta_5 = 0$ \\
Or, in more general form, $H_0: \beta_j  = 0$

$H_a: \beta_j \neq 0$ (at least one of the coefficients is not equal to zero)

\vspace{.25cm}
\textbf{P-value:}
To establish whether at least one predictor is a significant predictor in our logistic regression model, we should compare the full model to the reduced model. We can use \texttt{anova()} function to perform a likelihood ratio test.
\begin{verbatim}
	Model 1: choice ~ countries + sanctions
	Model 2: choice ~ 1
	Resid.  Df Resid. Dev Df Deviance  Pr(>Chi)    
	1      8494      11568                          
	2      8499      11783 -5  -215.15 < 2.2e-16 ***
\end{verbatim}

P-value \textbf{$< 2.2e^{-16}$ }is extremely small and close to zero, therefore, we have found strong evidence to \underline{reject} the null hypothesis (the additional parameters within the full model do not significantly improve the fit as compared to the reduced model) at the $\alpha$ level = 0.05.


	\end{enumerate}
	
	\item
	If any of the explanatory variables are significant in this model, then:
	\begin{enumerate}
		\item
		For the policy in which nearly all countries participate [160 of 192], how does increasing sanctions from 5\% to 15\% change the odds that an individual will support the policy? (Interpretation of a coefficient)
		\vspace{.25cm}


As our model is additive (not interactive), any $\beta$ change in  \texttt{sanctions} will have the same effect on the odds of an individual supporting the policy (\texttt{choice} = 1), regardless of the number of \texttt{countries}  participating in the international agreement, be it 20, 80, or 160.

So, we could calculate the effect of increasing sanctions from  5\% to 15\% as a difference between $\beta_4$ and $\beta_3$ as follows:

\lstinputlisting[language=R, firstline=62, lastline=63]{PS2MS.R} 

\begin{verbatim}
	-0.3251028
\end{verbatim}

\textbf{Interpretation:} On average, holding all the other variables constant, increasing sanctions from 5\% to 15\% is associated with a 0.3251028 decrease in the log odds of an individual supporting the policy.

\lstinputlisting[language=R, firstline=65, lastline=67]{PS2MS.R} 

\begin{verbatim}
 0.7224531
\end{verbatim}

\lstinputlisting[language=R, firstline=69, lastline=71]{PS2MS.R} 

\begin{verbatim}
	27.75469 
\end{verbatim}

Therefore, on average, holding all the other variables constant, increasing sanctions from 5\% to 15\% is associated with \( e^{-0.3251028} \approx 0.722 \approx 27.76\% \) decrease in the odds of an individual supporting the policy.

		\item
		What is the estimated probability that an individual will support a policy if there are 80 of 192 countries participating with no sanctions? \\
	\vspace{.25cm}
	
The formula for such estimation could be given by:
		
\[ P(Y_i = 1|X_i) = \frac{e^{\beta_0 + \beta_1 x_{\text{countries80}} + \beta_2 x_{\text{sanctionsnone}}}}{1 + e^{\beta_0 + \beta_1 x_{\text{countries80}} + \beta_2 x_{\text{sanctionsnone}}}} = \frac{1}{1 + e^{-(\beta_0 + \beta_1 x_{\text{countries80}} + \beta_2 x_{\text{sanctionsnone}})}} \]

Since we chose the "None" category for \texttt{sanctions} as a reference category, it is not included in the model, and we can assume that its coefficient equals to zero. 

Given that \( X_{\text{countries80}} = 1 \) and \( \beta_2 = 0 \),

\[ P(Y_i = 1|X_i) = \frac{e^{\beta_0 + \beta_1 \cdot 1 + 0 \cdot X_{\text{sanctionsnone}}}}{1 + e^{\beta_0 + \beta_1 \cdot 1 + 0 \cdot X_{\text{sanctionsnone}}}} = \frac{1}{1 + e^{-(\beta_0 + \beta_1)}} \]


	\lstinputlisting[language=R, firstline=73, lastline=76]{PS2MS.R} 
	\begin{verbatim}
		 0.5159191 
	\end{verbatim}

	\lstinputlisting[language=R, firstline=78, lastline=80]{PS2MS.R} 
	
		\begin{verbatim}
		0.5159191 
	\end{verbatim}

There is an estimated probability of approximately 0.52 = 52\% that an individual will support a policy if there are 80 of 192 countries participating with no sanctions.
	
		\item
		
		Would the answers to 2a and 2b potentially change if we included the interaction term in this model? Why? 
		\begin{itemize}
			\item Perform a test to see if including an interaction is appropriate.
		\end{itemize}
		
\textbf{First}, let's run an interactive logistic regression model: 

	\lstinputlisting[language=R, firstline=82, lastline=86]{PS2MS.R} 
	
\textit{	(Oops, it is a really long summary table -- please see the next page)}

	\begin{table}[H] \centering   \caption{}   \label{} \begin{tabular}{@{\extracolsep{5pt}}lc} \\[-1.8ex]\hline \hline \\[-1.8ex]  & \multicolumn{1}{c}{\textit{Dependent variable:}} \\ \cline{2-2} \\[-1.8ex] & choice \\ \hline \\[-1.8ex]  countries80 of 192 & 0.376$^{***}$ \\   & (0.106) \\   & \\  countries160 of 192 & 0.613$^{***}$ \\   & (0.108) \\   & \\  sanctions5\% & 0.122 \\   & (0.105) \\   & \\  sanctions15\% & $-$0.097 \\   & (0.108) \\   & \\  sanctions20\% & $-$0.253$^{**}$ \\   & (0.108) \\   & \\  countries80 of 192:sanctions5\% & 0.095 \\   & (0.152) \\   & \\  countries160 of 192:sanctions5\% & 0.130 \\   & (0.151) \\   & \\  countries80 of 192:sanctions15\% & $-$0.052 \\   & (0.152) \\   & \\  countries160 of 192:sanctions15\% & $-$0.052 \\   & (0.153) \\   & \\  countries80 of 192:sanctions20\% & $-$0.197 \\   & (0.151) \\   & \\  countries160 of 192:sanctions20\% & 0.057 \\   & (0.154) \\   & \\  Constant & $-$0.275$^{***}$ \\   & (0.075) \\   & \\ \hline \\[-1.8ex] Observations & 8,500 \\ Log Likelihood & $-$5,780.983 \\ Akaike Inf. Crit. & 11,585.970 \\ \hline \hline \\[-1.8ex] \textit{Note:}  & \multicolumn{1}{r}{$^{*}$p$<$0.1; $^{**}$p$<$0.05; $^{***}$p$<$0.01} \\ \end{tabular} \end{table} 

Even at a glance, we can see that the coefficients for all the interactive terms are not statistically significant, which can give us a notion that adding interaction to the model does not vastly contribute to explaining the variability in the response variable  \texttt{choice}. Let's perform a likelihood ratio test to support or reject this hypothesis:

	\lstinputlisting[language=R, firstline=89, lastline=91]{PS2MS.R} 
	
\begin{verbatim}
	Model 1: choice ~ countries + sanctions
	Model 2: choice ~ countries * sanctions
	Resid. Df Resid. Dev Df Deviance Pr(>Chi)
	1      8494      11568                     
	2      8488      11562  6   6.2928   0.3912
\end{verbatim}

Given the \textbf{p-value} of 0.3912, we have \underline{not} found evidence to reject the null hypothesis (adding an interaction to the model does not significantly improve its fit) at the $\alpha$ level = 0.05. This suggests that the association between the response variable and the predictor variables can be explained by the main effects of the predictors alone, without considering their interactions.
	
Furthermore, including the interaction terms would potentially change the answers both to 2(a) and 2(b):
\begin{itemize}
	\item \textbf{2(a)}: A given $\beta$ change in  \texttt{sanctions} may not have the same effect on the odds of an individual supporting the policy (\texttt{choice} = 1), as it may now be dependent on the specific number of \texttt{countries}  participating in the international agreement ('levels' of the other explanatory variable). However, the coefficient for the interactive term is not statistically significant.
	\item \textbf{2(b)}: Let's estimate the same probability (that an individual will support a policy if there are 80 of 192 countries participating with no sanctions) using the interactive model now, and then compare:
	
	\lstinputlisting[language=R, firstline=93, lastline=95]{PS2MS.R} 
	
	\begin{verbatim}
	0.5252101 
	\end{verbatim}
	
	Calculating the difference of two probabilities: 
	
		\lstinputlisting[language=R, firstline=98, lastline=98]{PS2MS.R} 
	
		\begin{verbatim}
	0.009291008 
	\end{verbatim}
	
Overall, the difference of two probabilities is quite minor, but it is still present.
	
\end{itemize}


	\end{enumerate}
	\end{enumerate}


\end{document}
